\documentclass[12pt]{exam}
\newcommand{\hwnumber}{10}
\newcommand{\hwname}{Memory Management and AVL Trees}

%%% begin hw-packages-and-macros.tex
\usepackage{amsmath}
\usepackage{xeCJK}% 调用 xeCJK 宏包

\usepackage[left=1in, right=1in, top=1in, bottom=1in]{geometry}
\usepackage{graphicx}
\usepackage{hyperref}
\usepackage{color}
\usepackage[all]{xy}
\usepackage{wrapfig}
\usepackage{fancyvrb}
\usepackage[T1]{fontenc}
\usepackage{listings}

\newcommand{\fillinlistings}{
    \lstset{ %
    language=C, numbers=left, numberstyle=\footnotesize, stepnumber=1,
    numbersep=4pt, showspaces=false, showstringspaces=false, tabsize=4,
    breaklines=true, breakatwhitespace=false,
    basicstyle=\normalsize\fontfamily{fvm}\selectfont, columns=flexible,
    mathescape=true, escapeinside={(*}{*)},
    morekeywords={alloc, allocarray, assert},
    otherkeywords={@requires, @ensures, @loop_invariant, @assert}
    }
}
\newcommand{\normallistings}{
    \lstset{ %
    language=C, numbers=left, numberstyle=\footnotesize, stepnumber=1,
    numbersep=4pt, showspaces=false, showstringspaces=false, tabsize=4,
    breaklines=true, breakatwhitespace=false,
    basicstyle=\footnotesize\fontfamily{fvm}\selectfont, columns=flexible,
    mathescape=true, escapeinside={(*}{*)},
    morekeywords={alloc, allocarray, assert},
    otherkeywords={@requires, @ensures, @loop_invariant, @assert}
    }
}

\newcommand{\semester}{Spring 2014}

\newcommand{\Cnought}{C$0$}
\newcommand{\modulo}{\ \texttt{\%}\ }
\newcommand{\lshift}{\ \texttt{<<}\ }
\newcommand{\rshift}{\ \texttt{>>}\ }
\newcommand{\cgeq}{\ \texttt{>=}\ }
\newtheorem{task}{Task}
\newtheorem{ectask}{Extra Credit Task}
\newtheorem{exercise}{Exercise}

\newcommand{\answerbox}[1]{
\begin{framed}
\hspace{5.65in}
\vspace{#1}
\end{framed}}

\pagestyle{head}


\headrule \header{\textbf{15-122 Written Homework \hwnumber}}{}{\textbf{Page
\thepage\ of \numpages}}

\pointsinmargin \printanswers

\setlength\answerlinelength{2in} \setlength\answerskip{0.3in}
%%% end hw-packages-and-macros.tex

\begin{document}

\addpoints
\begin{center}
\textbf{\large{15-122 : Principles of Imperative Computation, \semester
\\  \vspace{0.2in} Written Homework~\hwnumber
}}

 \vspace{0.2in}

 \large{Due before class: Thursday, April 3, 2014}
\end{center}

\vspace{0.5in}

\hbox to \textwidth{Name:\enspace\hrulefill}


\vspace{0.2in}

\hbox to \textwidth{Andrew ID:\enspace\hrulefill}

\vspace{0.2in}

\hbox to \textwidth{Recitation:\enspace\hrulefill}


\vspace{0.5in}

\noindent The written portion of this week's homework will give you some
practice working with C programming issues and AVL trees.
You can either type up your solutions or write them
\textit{neatly} by hand, and you should submit your work in class on the
due date just before lecture begins. Please remember to \textit{staple}
your written homework before submission.
\vspace{0.2in}

\begin{center}
\gradetable[v][questions]
\end{center}

\vspace{0.2in}
\begin{center}
  \Large{You must do this assignment in one of two ways and bring the stapled
         printout to the handin box on Thursday:

\bigskip
    1) Write your answers \textit{neatly} on a printout of this PDF.

\bigskip
    2) Use the TeX template at
    \url{http://www.cs.cmu.edu/afs/cs.cmu.edu/academic/class/15122-s14/www/theory10.tgz}
 }
\end{center}


\newpage
\begin{questions}

\question{\textbf{C Program Behavior}}

For each of the following problems, state what is wrong with the
code in one sentence. Do not just try to compile it and write down the
error message. (Some of these will compile without error, and some will even run and
produce output, but they all contain conceptual errors that may affect correctness.)
Read the code and explain what is being done wrong, conceptually.

\begin{parts}
\part[1]

\begin{verbatim}
#include <stdio.h>
#include <string.h>
int main() {
   ...
   char *w;
   strcpy(w,"C programming");      // copy string to w
   printf("%s\n", w);
   ...
   return 0;
}
\end{verbatim}

\begin{solution}
w声明为字符型指针,未初始化,即未分配空间,为悬挂指针。向悬挂指针拷贝字符串将导致段错误。

\end{solution}

\part[1]

\begin{verbatim}
#include <stdio.h>
#define MULT(X,Y) (X*Y)
int main() {
   ...
   int c = MULT(2+3,3+4);
   printf("(2+3)*(3+4) is = %d\n", c);
   ...
   return 0;
}
\end{verbatim}

\begin{solution}
宏定义错误,导致计算实际为2+3*3+4,存在二义性。
\end{solution}

\newpage
\part[1]

\begin{verbatim}
#include <stdlib.h>
#include "xalloc.h"
int main() {
   int *a = xmalloc(100);
   for (int i=0; i<100; i++)
      a[i]=i;
   ...
   free(a);
   return 0;
}
\end{verbatim}

\begin{solution}
a分配内存空间大小错误,循环赋值将超出抛出段错误异常。
且xmalloc非标准库。
\end{solution}


\part[1]
This code fragment shows a C function that is called from another function. It is
supposed to return the result only if no overflow occurs.
\textit{Hint:} You might
want to read the section on integers in the C0 to C tutorial
\url{http://c0.typesafety.net/tutorial/From-C0-to-C:-Basics.html#wiki-integers}.

\begin{verbatim}

#include <assert.h>
int oadd(int x, int y) {
   int result = x + y;
   if (x > 0 && y > 0) assert(result > 0);
   if (x < 0 && y < 0) assert(result < 0);
   return result;
}
\end{verbatim}

\begin{solution}
x+有可能导致有符号溢出,在C语言中属于Undefined Behavior
\end{solution}


\end{parts}


\newpage
\question{\textbf{Pass by reference using C}}

At various points in our C0 programming experience we had to use
somewhat awkward workarounds to deal with {\it functions that need to
return more than one value}. The address-of operator (\verb"&") in C gives us a
new way of dealing with this issue. In C, the expression \texttt{\&x} evaluates
to the address of the variable \texttt{x}.

\begin{parts}

\part[2]
Sometimes, a function needs to be able to both 1) signal whether it
can return a result, and 2) return that result if it is able to. Consider the
following function \texttt{parse\_string} that parses a string into an integer
if it is possible:

\vspace{0.1in}
\begin{verbatim}
bool parse(char *s, int *i);
// Returns true iff parse succeeds

void parse_string(char *s) {
  REQUIRES(s != NULL);
  int *i = xmalloc(sizeof(int));
  if (parse(s, i))
    printf("Success: %d.\n", *i);
  else
    printf("Failure.\n");
  free(i);
  return;
}
\end{verbatim}
\vspace{0.1in}

The \texttt{parse\_string} function relies on \texttt{parse} which both
sets \text{*i} to an integer equivalent to the integer pattern in \texttt{*s}
(if possible) and also returns a boolean value of true if the parse succeeds,
or false otherwise.

Using the address-of operator, rewrite the body of the \verb'parse_string'
function so that it does not heap-allocate, free, or leak any memory on the
heap. You may assume \verb"parse" has been implemented (its prototype
is given above).

\begin{solution}
\begin{verbatim}
void parse_string(char *s) {
  REQUIRES(s != NULL);
  int i;
  if (parse(s, &i))
    printf("Success: %d.\n", *i);
  else
    printf("Failure.\n");
  return;
}
\end{verbatim}
\end{solution}

\newpage
\part[2] In both C and C0, multiple values can be `returned' by bundling
them in a struct:

\begin{verbatim}
struct bundle { int x; int y; };
struct bundle *foo(int p) {
  ...
  struct bundle *A = xmalloc(sizeof(struct bundle));
  A->x = e1;     // first value to be returned
  A->y = e2;     // second value to be returned
  return A;      // return both values together as a struct
}
int main() {
  ...
  struct bundle *B = foo(p);
  int x = B->x;
  int y = B->y;
  free(B);
  ...
}
\end{verbatim}

Rewrite the declaration and the last few lines of the function \verb'foo',
as well as the snippet of \verb'main', to avoid heap-allocating, freeing, or
leaking any memory on the heap. The rest of the code (\verb"...") should continue
to behave exactly as it did before.

\begin{solution}

\vspace{15pt}
\begin{verbatim}
void foo(struct bundle * A, int p) {
  ...
  A->x = e1;
  A->y = e2;
  return;
}

int main() {
  ...
  struct bundle B;

  foo(&B, p);

  int x = B->x;

  int y = B->y;
  ...
}
\end{verbatim}
\end{solution}


\end{parts}


\newpage
\question{\textbf{AVL Trees.}}

\begin{parts}

\part[3] Draw the AVL trees that result after successively
inserting the following keys into an initially empty tree, in the
order shown:

\vspace{0.25in}
\begin{center}
\texttt{89, 79, 45, 58, 10, 63, 31}
\end{center}
\vspace{0.25in}

Show the tree after each insertion and subsequent
re-balancing (if any) is completed: the tree after the first element, \verb'89',
is inserted into an empty tree, then the tree after \verb'79' is
inserted into the first tree, and so on for a total of seven
trees. Make it clear what order the trees are in.

Be sure to maintain and restore the BST invariants and the additional
balance invariant required for an AVL tree after each insert.

\begin{solution}

    \includegraphics[scale=0.6]{avl.png}
\end{solution}

\part
Recall our definition for the height $h$ of a tree:
\begin{quote}
\textbf{The height of a tree is the maximum length of a path from the root to a leaf. So the empty tree has height 0, the tree with one node has height 1, and a balanced tree with three nodes has height 2.}
\end{quote}
The minimum number of nodes $m$ in a valid AVL tree is related to its
height. The goal of this question is to quantify this relationship.

\begin{subparts}
\subpart[2] Fill in the table below relating the variables $h$ and $m$:
\vspace{0.25in}

\begin{center}
\begin{tabular}{|c|c|}
 \hline   \hspace{2mm} $h$  \hspace{2mm} & \hspace{2mm}  $m$ \hspace{2mm} \\[5pt]
 \hline 0 & 0  \\[5pt]
 \hline 1 & 1  \\[5pt]
 \hline 2 & 2  \\[5pt]
 \hline 3 & 4  \\[5pt]
 \hline 4 & 7  \\[5pt]
 \hline 5 & 12  \\[5pt]
 \hline 6 & 20  \\[5pt]
 \hline
\end{tabular}
\end{center}

\vspace{0.25in}
\subpart[1]
Guided by the table in part (i), give an expression for $m$ as a function
of $h$.
Here's a hint: recall that the $n$th Fibonacci number $F(n)$ is defined by:
\begin{align*}
F(0) &= 0 \\
F(1) &= 1 \\
F(n) &= F(n-1) + F(n-2), \qquad n > 1
\end{align*}
You may find it useful to use the Fibonacci function $F(n)$ in your
answer.  Your answer \underline{does not} need to be a closed form
expression; it could be a recursive definition like the one for
$F(n)$.

\begin{solution}
  \begin{align*}
    M(0) &= 0 \\
    M(1) &= 1 \\
    M(n) &= M(n-1) + M(n-2) + 1, \qquad n > 1
    \end{align*}
\end{solution}

\subpart[1] Give a closed form expression for $M(h)$, the
\emph{maximum} number of nodes in a valid AVL tree of height $h$.

\begin{solution}
$M(h) = 2^{h} - 1$
\vspace{10pt}
\end{solution}

\end{subparts}
\end{parts}


\end{questions}
\end{document}
